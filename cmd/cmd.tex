% Options for packages loaded elsewhere
\PassOptionsToPackage{unicode}{hyperref}
\PassOptionsToPackage{hyphens}{url}
%
\documentclass[
  a4paper,
]{article}
\usepackage{amsmath,amssymb}
\usepackage{setspace}
\usepackage{iftex}
\ifPDFTeX
  \usepackage[T1]{fontenc}
  \usepackage[utf8]{inputenc}
  \usepackage{textcomp} % provide euro and other symbols
\else % if luatex or xetex
  \usepackage{unicode-math} % this also loads fontspec
  \defaultfontfeatures{Scale=MatchLowercase}
  \defaultfontfeatures[\rmfamily]{Ligatures=TeX,Scale=1}
\fi
\usepackage{lmodern}
\ifPDFTeX\else
  % xetex/luatex font selection
\fi
% Use upquote if available, for straight quotes in verbatim environments
\IfFileExists{upquote.sty}{\usepackage{upquote}}{}
\IfFileExists{microtype.sty}{% use microtype if available
  \usepackage[]{microtype}
  \UseMicrotypeSet[protrusion]{basicmath} % disable protrusion for tt fonts
}{}
\makeatletter
\@ifundefined{KOMAClassName}{% if non-KOMA class
  \IfFileExists{parskip.sty}{%
    \usepackage{parskip}
  }{% else
    \setlength{\parindent}{0pt}
    \setlength{\parskip}{6pt plus 2pt minus 1pt}}
}{% if KOMA class
  \KOMAoptions{parskip=half}}
\makeatother
\usepackage{xcolor}
\usepackage[margin=1in]{geometry}
\usepackage{longtable,booktabs,array}
\usepackage{calc} % for calculating minipage widths
% Correct order of tables after \paragraph or \subparagraph
\usepackage{etoolbox}
\makeatletter
\patchcmd\longtable{\par}{\if@noskipsec\mbox{}\fi\par}{}{}
\makeatother
% Allow footnotes in longtable head/foot
\IfFileExists{footnotehyper.sty}{\usepackage{footnotehyper}}{\usepackage{footnote}}
\makesavenoteenv{longtable}
\usepackage{graphicx}
\makeatletter
\def\maxwidth{\ifdim\Gin@nat@width>\linewidth\linewidth\else\Gin@nat@width\fi}
\def\maxheight{\ifdim\Gin@nat@height>\textheight\textheight\else\Gin@nat@height\fi}
\makeatother
% Scale images if necessary, so that they will not overflow the page
% margins by default, and it is still possible to overwrite the defaults
% using explicit options in \includegraphics[width, height, ...]{}
\setkeys{Gin}{width=\maxwidth,height=\maxheight,keepaspectratio}
% Set default figure placement to htbp
\makeatletter
\def\fps@figure{htbp}
\makeatother
\setlength{\emergencystretch}{3em} % prevent overfull lines
\providecommand{\tightlist}{%
  \setlength{\itemsep}{0pt}\setlength{\parskip}{0pt}}
\setcounter{secnumdepth}{-\maxdimen} % remove section numbering
\ifLuaTeX
\usepackage[bidi=basic]{babel}
\else
\usepackage[bidi=default]{babel}
\fi
\babelprovide[main,import]{catalan}
% get rid of language-specific shorthands (see #6817):
\let\LanguageShortHands\languageshorthands
\def\languageshorthands#1{}
\ifLuaTeX
  \usepackage{selnolig}  % disable illegal ligatures
\fi
\usepackage{bookmark}
\IfFileExists{xurl.sty}{\usepackage{xurl}}{} % add URL line breaks if available
\urlstyle{same}
\hypersetup{
  pdfauthor={@tofermos 2024},
  pdflang={ca-ES},
  hidelinks,
  pdfcreator={LaTeX via pandoc}}

\title{CMD}
\author{@tofermos 2024}
\date{}

\begin{document}
\maketitle

{
\setcounter{tocdepth}{2}
\tableofcontents
}
\setstretch{1.5}
\newpage

\renewcommand\tablename{Tabla}

\begin{center}\rule{0.5\linewidth}{0.5pt}\end{center}

\section{Objectiu de la sessió}\label{objectiu-de-la-sessiuxf3}

Aprendre a obrir el símbol del sistema (CMD) i fer servir ordres
bàsiques per navegar i gestionar fitxers.

\begin{center}\rule{0.5\linewidth}{0.5pt}\end{center}

\section{1. Obrir CMD}\label{obrir-cmd}

\begin{itemize}
\tightlist
\item
  Escriure \texttt{cmd} al menú Inici i prémer Enter
\item
  Prem \texttt{Win\ +\ R}, escriu \texttt{cmd} i prem Enter
\item
  Opcional: Botó dret sobre \texttt{cmd} i seleccionar ``Executar com a
  administrador''
\end{itemize}

\begin{center}\rule{0.5\linewidth}{0.5pt}\end{center}

\section{2. Ordres bàsiques}\label{ordres-buxe0siques}

\begin{longtable}[]{@{}ll@{}}
\toprule\noalign{}
Ordre & Explicació \\
\midrule\noalign{}
\endhead
\bottomrule\noalign{}
\endlastfoot
\texttt{dir} & Llista els fitxers i carpetes de la ubicació actual \\
\texttt{cd} & Canvia de directori (exemple: \texttt{cd\ Documents}) \\
\texttt{cd\ ..} & Pujar un nivell al directori anterior \\
\texttt{cls} & Neteja la pantalla \\
\texttt{exit} & Tanca la finestra del CMD \\
\end{longtable}

\begin{center}\rule{0.5\linewidth}{0.5pt}\end{center}

\section{3. Exercici pràctic bàsic}\label{exercici-pruxe0ctic-buxe0sic}

\begin{enumerate}
\def\labelenumi{\arabic{enumi}.}
\tightlist
\item
  Obrir CMD
\item
  Escriure \texttt{cd\ Desktop}
\item
  Crear una carpeta amb \texttt{mkdir\ Prova}
\item
  Entrar-hi amb \texttt{cd\ Prova}
\item
  Crear un fitxer amb \texttt{echo\ Hola\ \textgreater{}\ hola.txt}
\item
  Veure el contingut amb \texttt{dir}
\item
  Llegir el fitxer amb \texttt{type\ hola.txt}
\end{enumerate}

\begin{center}\rule{0.5\linewidth}{0.5pt}\end{center}

\section{4. Altres ordres útils}\label{altres-ordres-uxfatils}

\begin{longtable}[]{@{}ll@{}}
\toprule\noalign{}
Ordre & Funció \\
\midrule\noalign{}
\endhead
\bottomrule\noalign{}
\endlastfoot
\texttt{mkdir} & Crear carpeta nova \\
\texttt{del} & Esborrar fitxer (exemple: \texttt{del\ hola.txt}) \\
\texttt{rmdir} & Esborrar carpeta (ha d'estar buida) \\
\texttt{copy} & Copiar fitxer \\
\texttt{move} & Moure o reanomenar fitxer \\
\end{longtable}

\begin{center}\rule{0.5\linewidth}{0.5pt}\end{center}

\section{5. Utilitat actual del CMD}\label{utilitat-actual-del-cmd}

\subsection{1. Control avançat del
sistema}\label{control-avanuxe7at-del-sistema}

Permet accions fora de l'abast de la interfície gràfica: gestió
d'usuaris, neteja de sistema, configuració avançada.

\subsection{2. Formació per a entorns
professionals}\label{formaciuxf3-per-a-entorns-professionals}

És essencial en camps com administració de sistemes, desenvolupament i
ciberseguretat.

\subsection{3. Compatible amb scripts i
automatització}\label{compatible-amb-scripts-i-automatitzaciuxf3}

S'utilitza en molts processos d'instal·lació i configuració mitjançant
\texttt{.bat} o integració amb PowerShell.

\subsection{4. Aprenentatge estructurat}\label{aprenentatge-estructurat}

Ajuda a entendre la jerarquia del sistema d'arxius i fomenta el
pensament lògic i seqüencial.

\subsection{5. Eina educativa
accessible}\label{eina-educativa-accessible}

Permet fer pràctiques bàsiques sense necessitat d'instal·lar programari
addicional.

\end{document}
