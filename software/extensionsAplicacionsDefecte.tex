% Options for packages loaded elsewhere
\PassOptionsToPackage{unicode}{hyperref}
\PassOptionsToPackage{hyphens}{url}
%
\documentclass[
  a4paper,
]{article}
\usepackage{amsmath,amssymb}
\usepackage{setspace}
\usepackage{iftex}
\ifPDFTeX
  \usepackage[T1]{fontenc}
  \usepackage[utf8]{inputenc}
  \usepackage{textcomp} % provide euro and other symbols
\else % if luatex or xetex
  \usepackage{unicode-math} % this also loads fontspec
  \defaultfontfeatures{Scale=MatchLowercase}
  \defaultfontfeatures[\rmfamily]{Ligatures=TeX,Scale=1}
\fi
\usepackage{lmodern}
\ifPDFTeX\else
  % xetex/luatex font selection
\fi
% Use upquote if available, for straight quotes in verbatim environments
\IfFileExists{upquote.sty}{\usepackage{upquote}}{}
\IfFileExists{microtype.sty}{% use microtype if available
  \usepackage[]{microtype}
  \UseMicrotypeSet[protrusion]{basicmath} % disable protrusion for tt fonts
}{}
\makeatletter
\@ifundefined{KOMAClassName}{% if non-KOMA class
  \IfFileExists{parskip.sty}{%
    \usepackage{parskip}
  }{% else
    \setlength{\parindent}{0pt}
    \setlength{\parskip}{6pt plus 2pt minus 1pt}}
}{% if KOMA class
  \KOMAoptions{parskip=half}}
\makeatother
\usepackage{xcolor}
\usepackage[margin=1in]{geometry}
\usepackage{graphicx}
\makeatletter
\def\maxwidth{\ifdim\Gin@nat@width>\linewidth\linewidth\else\Gin@nat@width\fi}
\def\maxheight{\ifdim\Gin@nat@height>\textheight\textheight\else\Gin@nat@height\fi}
\makeatother
% Scale images if necessary, so that they will not overflow the page
% margins by default, and it is still possible to overwrite the defaults
% using explicit options in \includegraphics[width, height, ...]{}
\setkeys{Gin}{width=\maxwidth,height=\maxheight,keepaspectratio}
% Set default figure placement to htbp
\makeatletter
\def\fps@figure{htbp}
\makeatother
\setlength{\emergencystretch}{3em} % prevent overfull lines
\providecommand{\tightlist}{%
  \setlength{\itemsep}{0pt}\setlength{\parskip}{0pt}}
\setcounter{secnumdepth}{-\maxdimen} % remove section numbering
\ifLuaTeX
\usepackage[bidi=basic]{babel}
\else
\usepackage[bidi=default]{babel}
\fi
\babelprovide[main,import]{catalan}
% get rid of language-specific shorthands (see #6817):
\let\LanguageShortHands\languageshorthands
\def\languageshorthands#1{}
\ifLuaTeX
  \usepackage{selnolig}  % disable illegal ligatures
\fi
\usepackage{bookmark}
\IfFileExists{xurl.sty}{\usepackage{xurl}}{} % add URL line breaks if available
\urlstyle{same}
\hypersetup{
  pdfauthor={@tofermos 2024},
  pdflang={ca-ES},
  hidelinks,
  pdfcreator={LaTeX via pandoc}}

\title{APLICACIONS PER DEFECTE I ASSOCIACIÓ A TIPUS DE FITXERS}
\author{@tofermos 2024}
\date{}

\begin{document}
\maketitle

{
\setcounter{tocdepth}{2}
\tableofcontents
}
\setstretch{1.5}
\newpage

\renewcommand\tablename{Tabla}

\section{Introducció}\label{introducciuxf3}

El tema presetn tracta de les extensions de fitxers, l'associació a
aplicacions i les aplicacions per defecte en Windows 11.

\begin{center}\rule{0.5\linewidth}{0.5pt}\end{center}

\section{2 Les extensions de fitxers i les aplicacions per defecte a
Windows
11}\label{les-extensions-de-fitxers-i-les-aplicacions-per-defecte-a-windows-11}

\subsection{Què són les extensions de
fitxers?}\label{quuxe8-suxf3n-les-extensions-de-fitxers}

Les extensions de fitxers són sufixos que apareixen al final del nom
d'un fitxer, separats per un punt (.), i serveixen per identificar el
tipus de fitxer. Per exemple, \texttt{.txt} indica un fitxer de text
pla, \texttt{.jpg} és una imatge, i \texttt{.mp4} un fitxer de vídeo.
Aquestes extensions ajuden el sistema operatiu a saber amb quina
aplicació obrir cada fitxer.

\section{3 Associació de fitxers a
aplicacions}\label{associaciuxf3-de-fitxers-a-aplicacions}

Windows 11 utilitza aquestes extensions per associar els fitxers amb
aplicacions concretes. Quan fas doble clic sobre un fitxer, el sistema
busca l'aplicació assignada a aquella extensió i l'obre amb ella. Per
exemple, si tens associats els fitxers \texttt{.pdf} a Microsoft Edge,
tots els PDF s'obriran amb aquest navegador per defecte.

L'usuari pot canviar aquesta associació segons les seues preferències.
Això permet, per exemple, obrir imatges amb un editor diferent o
documents amb un altre lector de PDFs.

\subsection{Com es canvia l'aplicació per defecte a Windows
11?}\label{com-es-canvia-laplicaciuxf3-per-defecte-a-windows-11}

Windows 11 ofereix una manera senzilla de gestionar aquestes
associacions:

\begin{enumerate}
\def\labelenumi{\arabic{enumi}.}
\tightlist
\item
  \textbf{Obrir la configuració}: Fes clic al botó \emph{Inici} i
  selecciona \emph{Configuració} (⚙️).
\item
  \textbf{Aplicacions}: A la barra lateral esquerra, selecciona
  \emph{Aplicacions} i després \emph{Aplicacions per defecte}.
\item
  \textbf{Buscar per tipus de fitxer}: Pots cercar una extensió
  concreta, com \texttt{.mp3}, i triar amb quina aplicació vols obrir
  aquest tipus de fitxer.
\item
  \textbf{Canviar per aplicació}: També pots seleccionar una aplicació i
  veure quins tipus de fitxer té associats. Pots modificar les
  associacions aquí mateix.
\end{enumerate}

\section{4 Aplicacions recomanades per
defecte}\label{aplicacions-recomanades-per-defecte}

Windows 11 ve amb un conjunt d'aplicacions predeterminades:

\begin{itemize}
\tightlist
\item
  \textbf{Microsoft Edge} per a navegar per internet i obrir PDFs.
\item
  \textbf{Fotos} per veure imatges.
\item
  \textbf{Media Player} per a vídeos i música.
\item
  \textbf{Bloc de notes} o \textbf{WordPad} per a fitxers de text
  senzill.
\end{itemize}

Aquestes aplicacions es poden canviar fàcilment si prefereixes
alternatives com Google Chrome, VLC Media Player o Adobe Reader.

\section{5 Conclusió}\label{conclusiuxf3}

Entendre com funcionen les extensions de fitxers i com associar-les a
diferents aplicacions ens permet personalitzar millor l'experiència d'ús
del nostre ordinador. Windows 11 facilita aquesta tasca amb una
configuració clara i accessible per adaptar el sistema a les necessitats
de cada usuari.

\section{6 Activitat}\label{activitat}

\begin{itemize}
\tightlist
\item
  Prova instal·lar el Acrobat Reader i\ldots{}
\item
  assigna esta aplicació per defecte als fitxers PDF.
\item
  Mostra com es pot obrir un pdf amb una altra aplicació distinta.
\item
  Desinstal·la posteriorment el Acrobat Reader i\ldots{}
\item
  observa què passa. Continua sent l'aplicació per defecte?
\end{itemize}

\end{document}
